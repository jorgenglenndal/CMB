\documentclass{aa}  

\usepackage{graphicx}
\usepackage{txfonts}
\usepackage{hyperref}
\usepackage{xcolor}           % set colors

\hypersetup{ % this is just my personal choice, feel free to change things
    colorlinks,
    linkcolor={red!50!black},
    citecolor={blue!50!black},
    urlcolor={blue!80!black}}



\begin{document} 

   \title{Temporary title: CMB power spectrum}

   \author{J. A. Glenndal}

   \institute{Institute of Theoretical Astrophysics,  
                University of Oslo,  0315 Oslo,  Norway\\
              \email{j.a.glenndal@astro.uio.no}
             }

   \date{}

  \abstract{An abstract for the paper. Describe the paper. What is the paper about, what are the main results, etc.}

   \keywords{   cosmic microwave background  --
                large-scale structure of Universe
               }

   \maketitle

\section{Introduction}
Write an introduction here. Give context to the paper. Citations to relevant papers. You only need to do this in the end for the last milestone.

\section{Milestone I}
%Some introduction about what it is all about.
In this milestone we will look at the expansion history of a homogeneous and isotropic universe governed by the well known Friedmann equation \ref{Friedmann}.
The universe we consider consists of baryonic matter ($\Omega_b$), dark matter ($\Omega_\mathrm{CDM}$), radiation ($\Omega_\gamma$), neutrinos ($\Omega_\nu$)
and dark energy ($\Omega_\Lambda$), where $\Omega$ is the mass/energy density divided by the critical density ($\rho_c$). Since our goal in the end is to study the
cosmic microwave background (CMB), the homogeneous solution of the universe is of great interest. This is because the CMB is close to
being homogeneous with perturbations of order $10^{-5}$.
     


\subsection{Theory}
%The theory behind this milestone.
The parameters we use for our universe are given below.
%\vspace*{-1.5cm}

\begin{equation}
      \boxed{
   \begin{aligned}
      h &= 0.67, \\
      T_{\rm CMB 0} &= 2.7255\,K, \\
      N_{\rm eff} &= 3.046, \\
      \Omega_{\rm b 0} &= 0.05, \\
      \Omega_{\rm CDM 0} &= 0.267,\\
      \Omega_{k 0} &= 0, \\
      \Omega_{\nu 0} &= N_{\rm eff}\cdot \frac{7}{8}\left(\frac{4}{11}\right)^{4/3}\Omega_{\gamma 0}, \\
      \Omega_{\gamma 0} &= 2\cdot \frac{\pi^2}{30} \frac{(k_bT_{\rm CMB 0})^4}{\hbar^3 c^5} \cdot \frac{8\pi G}{3H_0^2},\\
      \Omega_{\Lambda 0} &= 1 - (\Omega_{k 0}+\Omega_{b 0}+\Omega_{\rm CDM 0}+\Omega_{\gamma 0}+\Omega_{\nu 0}),
   \end{aligned}}
\end{equation}

%\vspace*{0.5cm}
where the subscript 0 denotes today's value. $h$ is the small Hubble constant. More details can be found at.% \cite{ThomasYoung}.

The Friedmann equation is given by
\begin{equation}\label{Friedmann}
      H = H_0 \sqrt{(\Omega_{b0}+\Omega_{\rm CDM 0})a^{-3} + (\Omega_{\gamma 0} + \Omega_{\nu 0}) a^{-4} + \Omega_{k 0} a^{-2} + \Omega_{\Lambda 0}},
\end{equation}
where $a$ is the scale factor, $H = \frac{\dot{a}}{a}$ and $H_0$ is today's value of $H$. We will not use cosmic time ($t$) as our time variable.
Instead, we use $x=\ln a$ as our dimensionless time variable. This implies that $a = e^x$ for conversion. Since $a(t=0) =0$ and $a(t=t_0) = 1$, where $t_0$ is time today
we get $t=0 \iff x = -\infty$ and $t=t_0 \iff x = 0$. The cosmic time can be found from the differential equation 
\begin{equation}\label{cosmic_time_differential_equation}
      \frac{dt}{dx} = \frac{1}{H}.
\end{equation}


The $\Omega$s can be expressed as
\begin{equation}
      \hspace*{2cm}
   \begin{aligned}
      \Omega_{k}(a) &= \frac{\Omega_{k0}}{a^2H(a)^2/H_0^2}\\
      \Omega_{\rm CDM}(a) &= \frac{\Omega_{\rm CDM 0}}{a^3H(a)^2/H_0^2} \\
      \Omega_b(a) &= \frac{\Omega_{b 0}}{a^3H(a)^2/H_0^2} \\
      \Omega_\gamma(a) &= \frac{\Omega_{\gamma 0}}{a^4H(a)^2/H_0^2} \\
      \Omega_{\nu}(a) &= \frac{\Omega_{\nu 0}}{a^4H(a)^2/H_0^2} \\
      \Omega_{\Lambda}(a) &= \frac{\Omega_{\Lambda 0}}{H(a)^2/H_0^2}.
   \end{aligned}
\end{equation}


\subsection{Implementation details}
Something about the numerical work.

\subsection{Results}
Show and discuss the results.

\section{Milestone II}
Some introduction about what it is all about.

\subsection{Theory}
The theory behind this milestone.

\subsection{Implementation details}
Something about the numerical work.

\subsection{Results}
Show and discuss the results.

\section{Milestone III}
Some introduction about what it is all about.

\subsection{Theory}
The theory behind this milestone.

\subsection{Implementation details}
Something about the numerical work.

\subsection{Results}
Show and discuss the results.

\section{Milestone IV}
Some introduction about what it is all about.

\subsection{Theory}
The theory behind this milestone.

\subsection{Implementation details}
Something about the numerical work.

\subsection{Results}
Show and discuss the results.

\section{Conclusions}

Write a short summary and conclusion in the end. 

\begin{acknowledgements}
      I thank my mom for financial support!
\end{acknowledgements}

%\bibliographystyle{plain} % We choose the "plain" reference style
%\bibliography{refs} % Entries are in the refs.bib file

%\begin{thebibliography}{}
%      \bibitem[1966]{baker} Baker, N. 1966,
%      in Stellar Evolution,
%      ed.\ R. F. Stein,\& A. G. W. Cameron
%      (Plenum, New York) 333
%
%      \bibitem[2023]{winther} Winther, H.A. 2023,
%      %\href{https://cmb.wintherscoming.no/milestone1.php}
%\end{thebibliography}

\end{document}